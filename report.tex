\documentclass{article}
\usepackage{graphicx}
\usepackage{hyperref}
\usepackage{listings}
\usepackage{xcolor}
\usepackage{geometry}
\geometry{a4paper, margin=1in}
\setlength{\parindent}{0pt}
\setlength{\parskip}{1em}

\title{Software Design Report}
\author{Project Team}
\date{\today}

\begin{document}

\maketitle

\section{Introduction}
This document provides an overview of the design decisions, challenges faced, program structure, edge cases, and error handling of our software system. The document includes a diagram illustrating key files, data structures, and functions. Additionally, the report contains links to the GitHub repository and a video demonstration.

\section{Design Decisions}
Our software is structured as a command-line spreadsheet application that supports arithmetic operations, functions like SUM, MIN, MAX, and dependency tracking. The key design decisions include:
\begin{itemize}
    \item Using a graph-based structure to handle cell dependencies.
    \item Implementing an AVL tree to maintain dependencies efficiently.
    \item Enabling a user-friendly interface with interactive navigation.
    \item Using a modular approach with separate files for display, computation, and graph logic.
    \item Incorporating robust error handling mechanisms to handle invalid user input and system errors.
    \item Ensuring efficient memory management for large datasets.
\end{itemize}

\section{Challenges Faced}
During the development, we encountered several challenges:
\begin{itemize}
    \item Implementing a cycle detection mechanism in formula dependencies.
    \item Managing memory efficiently to handle large spreadsheets.
    \item Ensuring correct parsing of user inputs with a robust error-handling mechanism.
    \item Optimizing performance while recalculating cell values upon dependency updates.
    \item Designing a user-friendly display and navigation system within the constraints of a terminal-based UI.
\end{itemize}

\section{Program Structure}
Our software consists of the following key files:
\begin{itemize}
    \item \textbf{1.c} - The main entry point of the program.
    \item \textbf{display.c, display.h} - Handles rendering of the spreadsheet.
    \item \textbf{Functions.c, Functions.h} - Implements arithmetic and aggregate functions.
    \item \textbf{Graph.c, Graph.h} - Manages the dependency graph.
    \item \textbf{Parser.c, Parser.h} - Parses user commands.
    \item \textbf{Makefile} - Provides automated build commands.
\end{itemize}

\begin{figure}[h]
    \centering
    
    \caption{Software Structure}
    \label{fig:diagram}
\end{figure}

\section{Edge Cases and Error Handling}
Our test suite covers various edge cases, including:
\begin{itemize}
    \item Handling division by zero.
    \item Detecting and preventing circular dependencies.
    \item Managing large spreadsheets efficiently.
    \item Correctly parsing invalid inputs and providing meaningful error messages.
    \item Handling negative numbers and large integer values.
    \item Ensuring correctness in dependency-based calculations with deeply nested formulas.
    \item Validating correct input formats and rejecting malformed expressions.
\end{itemize}

\section{Performance Considerations}
To ensure efficiency and responsiveness, we employed:
\begin{itemize}
    \item AVL trees for efficient dependency tracking and updates.
    \item A queue-based topological sorting approach to avoid redundant calculations.
    \item Caching mechanisms to optimize repeated computations and minimize redundant work.
    \item Asynchronous computation for large formula evaluations to prevent UI lag.
\end{itemize}

\section{Build and Execution}
The software can be compiled and executed using the provided Makefile:
\begin{lstlisting}
make        # Compiles and generates the executable
make test   # Runs all test cases
make report # Generates report.pdf from LaTeX source
\end{lstlisting}

To run the application:
\begin{lstlisting}
./sheet 10 10  # Launches a 10x10 spreadsheet
\end{lstlisting}

\section{Conclusion}
Our software successfully implements a functional spreadsheet system with arithmetic operations, functions, and dependency management. The modular architecture ensures maintainability and scalability, while robust error handling enhances usability. Future improvements could include a GUI-based interface, additional functions, and integration with external data sources.

\section{Links}
\textbf{Video Demo}: \href{https://example.com/demo}{Watch Here}\\
\textbf{GitHub Repository}: \href{https://github.com/example/repo}{View Here}

\end{document}
